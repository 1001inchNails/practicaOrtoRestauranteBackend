\documentclass[12pt]{article}
\usepackage[spanish]{babel}
\usepackage[utf8]{inputenc}
\usepackage{graphicx}
\usepackage{xcolor}
\usepackage{geometry}
\usepackage{amsmath}
\usepackage[section]{placeins}
\usepackage{hyperref}
\hypersetup{
    colorlinks = true,
    linkcolor = blue,
    urlcolor = cyan
}

\geometry{a4paper, margin=2.5cm}

\title{Documentación de API de Gestión de Restaurante}
\author{Daniel Calvar Cruz}
\date{\today}

\begin{document}

\maketitle

\tableofcontents
\newpage

\section{Introducción}
\hyperlink{anchor-indice}{\textbf{Volver}}\\

Esta API proporciona un sistema completo para la gestión de un restaurante, incluyendo el manejo de productos del menú y pedidos de mesas.\\

La API está construida con Express.js y utiliza MongoDB como base de datos.

\section{Estructura}

\subsection{Rutas Principales}
La API está organizada en tres módulos principales:

\begin{itemize}
    \item \textbf{POST}: Operaciones de creación
    \item \textbf{GET}: Operaciones de lectura
    \item \textbf{DELETE}: Operaciones de eliminación
\end{itemize}

\subsection{Modelos de Datos}

\subsubsection{Modelo Producto}
\begin{verbatim}
const ProductoSchema = new mongoose.Schema({
  id: { type: Number },
  nombre: { type: String },
  precio: { type: Number },
  descripcion: { type: String }
});
\end{verbatim}

\subsubsection{Modelo Mesa}
\begin{verbatim}
const Mesa1Schema = new mongoose.Schema({
  pedidos: { type: Array }
});

"pedidos": array de objetos Producto.
\end{verbatim}

\clearpage

\section{Endpoints}
\hyperlink{anchor-indice}{\textbf{Volver}}\\

\subsection{Endpoints POST}

\subsubsection{POST /post/createobjetomenu}
\textbf{Descripción:} Crea un nuevo producto en el menú.\\

\textbf{Body:}
\begin{verbatim}
{
  "nombre": "string",
  "precio": "number",
  "descripcion": "string"
}
\end{verbatim}

\textbf{Respuestas:}
\begin{itemize}
    \item \textbf{200}: Documento creado exitosamente
    \item \textbf{400}: Error en los datos proporcionados
    \item \textbf{500}: Error interno del servidor
\end{itemize}

\subsubsection{POST /post/mandarpedidoamesa}
\textbf{Descripción:} Envía un pedido a una mesa específica.\\

\textbf{Body:}
\begin{verbatim}
{
  "mesa": "string",
  "pedidos": [
    {
      "id": "number",
      "nombre": "string",
      "precio": "number",
      "descripcion": "string"
    }
  ]
}
"mesa": Mesa1, Mesa2, etc.
"pedidos": array de productos.
\end{verbatim}

\textbf{Respuestas:}
\begin{itemize}
    \item \textbf{200}: Pedidos creados exitosamente
    \item \textbf{400}: Error en los datos del pedido
    \item \textbf{500}: Error interno del servidor
\end{itemize}

\clearpage

\subsection{Endpoints GET}
\hyperlink{anchor-indice}{\textbf{Volver}}\\

\subsubsection{GET /get/readallmenus}
\textbf{Descripción:} Obtiene todos los productos del menú.\\

\textbf{Parámetros:} Ninguno\\

\textbf{Respuestas:}
\begin{itemize}
    \item \textbf{200}: Información encontrada con datos
    \item \textbf{404}: Documentos no encontrados
    \item \textbf{500}: Error interno del servidor
\end{itemize}

\subsubsection{GET /get/readmesa}
\textbf{Descripción:} Obtiene los pedidos de una mesa específica.\\

\textbf{Parámetros:}
\begin{itemize}
    \item \texttt{mesaId}: Identificador de la mesa (ej: "Mesa1")
\end{itemize}

\textbf{Respuestas:}
\begin{itemize}
    \item \textbf{200}: Información encontrada
    \item \textbf{400}: Modelo no encontrado
    \item \textbf{404}: No hay datos en la colección
    \item \textbf{500}: Error en proceso de búsqueda
\end{itemize}

\clearpage

\subsection{Endpoints DELETE}
\hyperlink{anchor-indice}{\textbf{Volver}}\\

\subsubsection{DELETE /delete/deletemesa}
\textbf{Descripción:} Elimina todos los pedidos de una mesa específica.\\

\textbf{Parámetros:}
\begin{itemize}
    \item \texttt{mesaId}: Identificador de la mesa (ej: "Mesa1")
\end{itemize}

\textbf{Respuestas:}
\begin{itemize}
    \item \textbf{200}: Información borrada exitosamente
    \item \textbf{400}: Modelo no encontrado
    \item \textbf{500}: Error en proceso de borrado
\end{itemize}

\clearpage

\section{Flujos de Trabajo}
\hyperlink{anchor-indice}{\textbf{Volver}}\\

\subsection{Agregar Producto al Menú}
\begin{enumerate}
    \item El sistema genera automáticamente un ID incremental
    \item Valida que los datos del producto sean correctos
    \item Guarda el producto en la colección Producto
\end{enumerate}

\subsection{Realizar un Pedido}
\begin{enumerate}
    \item Valida que los productos existan en la base de datos
    \item Verifica la estructura de los datos del pedido
    \item Guarda el pedido en la mesa correspondiente
\end{enumerate}

\clearpage

\section{Manejo de Errores}
\hyperlink{anchor-indice}{\textbf{Volver}}\\

La API utiliza un sistema consistente de respuestas:

\begin{verbatim}
{
  "type": "success|failure",
  "message": "string descriptivo",
  "data": "object (opcional)"
}
\end{verbatim}

\section{Ejemplos de Uso}

\subsection{Ejemplo: Crear Producto}
\begin{verbatim}
POST /post/createObjetoMenu
{
  "nombre": "Hamburguesa Clásica",
  "precio": 12.99,
  "descripcion": "Hamburguesa con queso, lechuga y tomate"
}
\end{verbatim}

\subsection{Ejemplo: Hacer Pedido}
\begin{verbatim}
POST /post/mandarPedidoAMesa
{
  "mesa": "Mesa1",
  "pedidos": [
    {
      "id": 1,
      "nombre": "Hamburguesa Clásica",
      "precio": 12.99,
      "descripcion": "Hamburguesa con queso, lechuga y tomate"
    }
  ]
}
\end{verbatim}

\clearpage

\section{Consideraciones Técnicas}
\hyperlink{anchor-indice}{\textbf{Volver}}\\

\begin{itemize}
    \item Los IDs de productos se generan automáticamente de forma incremental
    \item Las mesas deben estar predefinidas en los modelos (Mesa1, Mesa2, etc.)
    \item Todos los precios deben ser números válidos
    \item La API valida la existencia de productos antes de crear pedidos
\end{itemize}

\section{Códigos de Estado HTTP}

\begin{itemize}
    \item \textbf{200 OK}: Operación exitosa
    \item \textbf{400 Bad Request}: Error en los datos enviados
    \item \textbf{404 Not Found}: Recurso no encontrado
    \item \textbf{500 Internal Server Error}: Error del servidor
\end{itemize}

\section{Stack tecnológico}

\begin{itemize}
    \item \textbf{Versión Node.js:} 22.14
    \item \textbf{Base de datos:} MongoDB Atlas
    \item \textbf{Dependencias NPM:}
    \begin{itemize}
        \item \textbf{cors:} 2.8.5
        \item \textbf{dotenv:} 16.4.7
        \item \textbf{express:} 4.21.1
        \item \textbf{mongodb:} 6.12.0
        \item \textbf{mongoose:} 8.13.2
        \item \textbf{nodemon:} 3.1.10  
    \end{itemize}
\end{itemize}

\section{Variables de entorno (.env)}

    \begin{itemize}
    \item \texttt{MONGO\_URI}: URI de conexión a MongoDB Atlas o local.
    \item \texttt{PORT}: Puerto en el que se ejecutará el servidor.
    \item \texttt{DDBB\_NAME}: Nombre de la base de datos en Mongo Atlas.
\end{itemize}

\end{document}